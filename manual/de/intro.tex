Traverso ist ein Mehrspur-Aufnahme- und Editierprogramm für GNU/Linux, das besonderen Wert auf eine intuitive, einfach, und vor allem effiziente Bedienoberfläche legt. Zur Zeit unterstützt das Programm die Aufnahme von beliebig vielen Spuren (nur durch die Möglichkeiten der Hardware begrenzt), grundlegende Mischfunktionen, das brennen von CDs, sowie den Export der Projekte in viele bekannte Audioformate. Die Bearbeitung der Audiosignale erfolgt durchgängig in 32 Bit Flie"skomma-Genauigkeit, wodurch ein Höchstma"s an Klangqualität auch nach intensiver Bearbeitung erhalten bleibt.

Die Bedienoberfläche basiert auf einem kontextsensitiven Interaktionskonzept, das durchgehend Kombinationen von Maus- und Tastenaktionen zur Bearbeitung von graphischen Objekten verwendet. Dies erlaubt eine vergleichsweise flexible und effiziente Arbeitsweise, und eröffnet zudem Bedienmöglichkeiten, die über normale Tastenkombinationen hinausgehen. Die Maus muss lediglich über einem Objekt plaziert werden, und schon erhält man direkten Zugriff auf alle verfügbaren Funktionen über verschiedene Tasten auf der Tastatur. Da das Objekt unter dem Mauszeiger automatisch ausgewählt wird, nennen wir dieses Konzept ,,weiche Selektion''.

\section{Lizenz}
Traverso ist freie Software. Sie können es unter den Bedingungen der GNU General Public License, wie von der Free Software Foundation veröffentlicht, weitergeben und/oder modifizieren, entweder gemäß Version 2 der Lizenz oder (nach Ihrer Option) jeder späteren Version. 

Die Veröffentlichung dieses Programms erfolgt in der Hoffnung, daß es Ihnen von Nutzen sein wird, aber OHNE IRGENDEINE GARANTIE, sogar ohne die implizite Garantie der MARKTREIFE oder der VERWENDBARKEIT FÜR EINEN BESTIMMTEN ZWECK. Details finden Sie in der GNU General Public License.

Sie sollten ein Exemplar der GNU General Public License zusammen mit diesem Programm erhalten haben. Falls nicht, schreiben Sie an die Free Software Foundation, Inc., 51 Franklin St, Fifth Floor, Boston, MA 02110, USA.

\section{Motivation}
Ein Grund für die Einführung der weichen Selektion war unser Glaube an die Überlegenheit des Konzepts gegenüber herkömmlichen Maus-lastigen Bedienkonzepten in bezug auf Effizienz, Geschwindigkeit und Ergonomie. Wie immer wenn etwas mit existierenden Standards bricht, entfaltet sich das volle Potential erst nach einer kurzen Lernphase. Um alte Gewohnheiten zu überwinden und das neue Konzept vollständig zu internalisieren ist sogar noch mehr Zeit nötig. Betrachten wir deshalb die verschiedenen Konzepte an einem einfachen Beispiel: Eine Spur soll auf Stumm (\textit{Mute}) oder Solo gestellt werden.

\minisec{Der ,,analoge Weg''}
Analoge Aufnahmestudios sind normalerweise mit gro"sen Mischpulten ausgestattet, die eine Unmenge an Dreh-, Schiebereglern und Knöpfen aufweisen. Um einen Kanal auf ,,Solo'' oder ..Stumm'' (,,Mute'') zu schalten, muss man erst einmal den richtigen Kanal und dann den richtigen Knopf finden, dann erhält man jedoch direkten Zugriff auf die gewünschte Funktion. Was bei einem Kanal noch einfach ist, wird mit einer grö"seren Anzahl Kanäle schnell zu einer Fingerübung. Den direkten Zugriff auf jede Funktion erkauft man sich in diesem Fall durch eine enorme Anzahl Bedienelemente, wodurch analoge Mischpulte zu gro"sen und komplexen elektronischen Geräten werden.

\minisec{Der ,,digitale Weg''}
Herkömmliche digitale Audio-Workstations (DAW) bilden normalerweise die Funktionen einer analogen Konsole auf dem Bildschirm ab. Entsprechend zahlreich sind die Bedienelemente, die unter Umständen sogar noch in Menüs aufgeteilt werden. Will man eine Funktion ausführen, muss also der Knopf gesucht und gedrückt werden. Aufgrund der begrenzten Platzverhältnisse auf dem Bildschirm und der Grö"se des Monitors fallen die Knöpfe oft sehr klein aus, weshalb Fehlklicks -- je nach Fähigkeiten des Benutzers mehr oder weniger -- häufig vorkommen. Zur Geduldsprobe wird solch eine Operation spätestens, wenn die Operation auf mehrere Kanäle angewandt werden soll. Aufgrund der schlechteren Zielgenauigkeit fällt dies in der Regel deutlich schwerer als auf einem analogen Mischpult.

\minisec{Der ,,Traverso-Weg''}
In Traverso wird der Mauszeiger über dem Objekt der Wahl positioniert. Die gewünschte Funktion wird dann über die Tastatur aufgerufen, z.\,B. ,,U'' für ,,Stumm'', oder ,,O'' für ,,Solo''. Wer das Zehnfinger-System beherrscht, findet die Taste in der Regel blind, und das graphische Objekt, das anvisiert werden muss, ist in den meisten Fällen grö"ser als ein Knopf (z.\,B. ein Audio-Clip, eine Spur im Track Panel etc.). Eine hohe Genauigkeit beim Zielen ist also nicht erforderlich.

Aber was ist nun der gro"se Unterschied zwischen dem digitalen und dem Traverso-Weg? Unserer Erfahrung nach vermittelt die Bedienung eines PCs mit der Maus immer ein Gefühl von ,,Fernsteuerung''. Man bewegt die Maus auf der Tischplatte, und steuert damit den Zeiger auf dem Bildschirm. Ein direkter Kontakt zwischen Finger und Bedienelement existiert nicht. Ein vergleichbares Beispiel wäre etwa, wenn man die Knöpfe auf einem analogen Mischpult nur mit einem Zeigestab oder Lineal bedienen dürfte. Das wäre sehr ermüdend und würde ein hohes Ma"s an Zielgenauigkeit erfordern, und vermutlich zu vielen Fehlklicks führen. Einen echten Knopf direkt mit den Fingern zu drücken ist dagegen viel ergonomischer und intuitiver. In Traveso versuchten wir deshalb ein Konzept zu finden, das es erlaubt, Funktionen über echte Knöpfe zu erreichen, ohne spezielle Hardware-Erweiterungen anschaffen zu müssen. Was liegt also näher als die Tastatur des PCs dazu zu verwenden? Die Maus wird also lediglich dazu verwendet, das Objekt, dass die Funktion empfangen soll, zu identifizieren. Die Funktion selbst wird dann von der linken Hand auf der Tastatur ausgeführt. Zu allem Überfluss können wir die meisten der 104 Tasten sogar mehrfach mit Funktionen belegen, was dem Benutzer direkten Zugriff auf eine riesige Anzahl Operationen erlaubt. Das Konzept ist eng verwandt mit der Bedienung von einigen Computerspielen (z.\,B. ,,First-person shooters''), die ein Maximum an Effizienz und direktem Zugriff auf Funktionen (gehen, rennen, schie"sen, ducken\dots) und Ausrüstungsgegenstände erfordern. Zusammengefasst bietet das Konzept der weichen Selektion also folgende Vorteile:

\begin{itemize}
 \item Die Distanzen zwischen den Tasten auf der Tastatur sind meist kürzer als zwischen den Knöpfen auf dem Bildschirm.
 \item Die Trefferquote für eine bestimmte Taste ist viel höher als für einen Knopf auf dem Bildschirm.
 \item Die Verwendung beider Hände ermöglicht eine schnellere Bedienung, entlastet die Maus-Hand, und führt zu einer weniger ermüdenden Arbeitsweise.
 \item Die Bedienung fühlt sich ,,direkter'' an, das Gefühl der ,,Fernbedienung'' wird deutlich reduziert.
 \item Direkter Zugriff auf viele Funktionen ermöglicht weniger und flachere Menüstrukturen.
 \item Die Tasten können blind gefunden werden, wodurch man schneller arbeiten kann und weniger Details auf dem Bildschirm erkennen muss.
\end{itemize}

Diese Vorteile erkauft man sich jedoch mit einer deutlich steileren Lernkurve, besonders ganz am Anfang. Da die Tasten nicht mit ihrer Funktion angeschrieben sind, muss man sie auswendig lernen. Eine sinnvolle Tastaturbelegung, die sich einerseits am Buchstaben der Taste orientiert, und zudem bemüht ist, ähnliche Funktionen möglichst immer auf ein- und dieselbe Taste zu legen, erleichtert jedoch das auswendig lernen, und schon nach kurzer Zeit werden die wichtigsten Funktionen genauso automatisch ausgeführt wie ,,Strg+C'' für ,,kopieren'' und ,,Strg+V'' für ,,einfügen''. Als letztes Beispiel für die Bedeutung von direktem Zugriff auf wichtige Funktionen sei das Bremspedal eines Autos aufgeführt. Niemand würde auf die Idee kommen, die Funktion zum bremsen im Bordcomputer-Menü ,,Geschwindigkeit'', Untermenü ,,Bremsen'', Untermenü ,,Bremsdruck ändern'' zu verstecken, oder?

