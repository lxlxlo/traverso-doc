Traverso ist ein Mehrspur-Aufnahme- und Editierprogramm für GNU/Linux, das besonderen Wert auf eine intuitive, einfach, und vor allem effiziente Bedienoberfläche legt. Zur Zeit unterstützt das Programm die Aufnahme von beliebig vielen Spuren (nur durch die Möglichkeiten der Hardware begrenzt), grundlegende Mischfunktionen, das brennen von CDs, sowie den Export von Projekten in viele bekannte Audioformate. Die Bearbeitung der Audiosignale erfolgt durchgängig in 32 Bit Flie"skomma-Genauigkeit, wodurch auch nach intensiver Bearbeitung ein Höchstma"s an Klangqualität erhalten bleibt.

Die Bedienoberfläche basiert auf einem kontextsensitiven Interaktionskonzept, das durchgehend Kombinationen von Maus- und Tastenaktionen zur Bearbeitung von graphischen Objekten verwendet. Dies erlaubt eine vergleichsweise flexible und effiziente Arbeitsweise, und eröffnet zudem Bedienmöglichkeiten, die über normale Tastenkombinationen hinausgehen. Die Maus muss lediglich über einem Objekt plaziert werden um über die Tastatur direkten Zugriff auf alle verfügbaren Funktionen zu erhalten. Da das Objekt unter dem Mauszeiger automatisch ausgewählt wird, wird dieses Konzept auch ,,weiche Selektion'' genannt.

\section{Lizenz}
Traverso ist freie Software. Sie können es unter den Bedingungen der GNU General Public License, wie von der Free Software Foundation veröffentlicht, weitergeben und/oder modifizieren, entweder gemäß Version 2 der Lizenz oder (nach Ihrer Option) jeder späteren Version. 

Die Veröffentlichung dieses Programms erfolgt in der Hoffnung, daß es Ihnen von Nutzen sein wird, aber OHNE IRGENDEINE GARANTIE, sogar ohne die implizite Garantie der MARKTREIFE oder der VERWENDBARKEIT FÜR EINEN BESTIMMTEN ZWECK. Details finden Sie in der GNU General Public License.

Sie sollten ein Exemplar der GNU General Public License zusammen mit diesem Programm erhalten haben. Falls nicht, schreiben Sie an die Free Software Foundation, Inc., 51 Franklin St, Fifth Floor, Boston, MA 02110, USA.

\section{Motivation}
Ein Grund für die Einführung der weichen Selektion war unser Glaube an die Überlegenheit des Konzepts gegenüber herkömmlichen mauslastigen Bedienkonzepten, vor allem in bezug auf Effizienz, Geschwindigkeit und Ergonomie. Wie immer wenn etwas mit existierenden Standards bricht entfaltet sich das volle Potential auch hier erst nach einer kurzen Lernphase. Um alte Gewohnheiten zu überwinden und das neue Konzept vollständig zu internalisieren ist sogar noch mehr Zeit nötig. Betrachten wir deshalb die verschiedenen Konzepte an einem einfachen Beispiel: Eine Spur soll auf Stumm (\textit{Mute}) oder Solo gestellt werden.

\minisec{Der ,,analoge Weg''}
Analoge Aufnahmestudios sind normalerweise mit gro"sen Mischpulten ausgestattet, die mit einer Unmenge an Dreh-, Schiebereglern und Knöpfen ausgestattet sind. Um einen Kanal auf ,,Solo'' oder ,,Stumm'' (,,Mute'') zu schalten, muss man erst einmal den richtigen Kanal und dann den richtigen Knopf finden, dann erhält man jedoch direkten Zugriff auf die gewünschte Funktion. Was bei einer Spur noch einfach ist, wird bei einer grö"seren Anzahl Kanäle schnell zu einer Fingerübung. Den direkten Zugriff auf jede Funktion erkauft man sich in diesem Fall durch eine enorme Anzahl Bedienelemente, weshalb analoge Mischpulte in der Regel komplexe elektronische Geräte mit entsprechend gro"sem Platzbedarf sind.

\minisec{Der ,,digitale Weg''}
Herkömmliche digitale Audio-Workstations (DAWs) bilden normalerweise die Funktionen einer analogen Konsole auf dem Bildschirm ab. Entsprechend zahlreich sind die Bedienelemente, die unter Umständen sogar noch in Menüs verborgen werden. Will man eine Funktion ausführen, muss ein Knopf gesucht und gedrückt werden. Aufgrund der begrenzten Platzverhältnisse auf dem Bildschirm und der Grö"se des Monitors fallen die Knöpfe oft sehr klein aus, weshalb Fehlklicks -- je nach Fähigkeiten des Benutzers mehr oder weniger -- häufig vorkommen. Zur Geduldsprobe wird solch eine Operation spätestens, wenn die Operation auf mehrere Kanäle angewandt werden soll. Aufgrund der schlechteren Zielgenauigkeit fällt dies oft deutlich schwerer als auf einem analogen Mischpult.

\minisec{Der ,,Traverso-Weg''}
Um die Vorteile der analogen und digitalen Bearbeitung zu kombinieren werden oft zusätzliche Eingabegeräte verwendet, welche die Steuerung der DAW-Software über reale Knöpfe und Schieberegler ermöglicht. Einen ähnlichen Ansatz, jedoch ohne auf zusätzliche Hardware angewiesen zu sein, verfolgt Traverso mit dem Konzept der weichen Selektion.

In Traverso wird der Mauszeiger über dem Objekt der Wahl positioniert. Die gewünschte Funktion wird dann über die Tastatur aufgerufen, z.\,B. ,,U'' für ,,Stumm'', oder ,,O'' für ,,Solo''. Wer das Zehnfinger-System beherrscht, findet die Taste in der Regel blind, und das graphische Objekt, das anvisiert werden muss, ist in den meisten Fällen grö"ser als ein Knopf (z.\,B. ein Audio-Clip, eine Spur im Track Panel etc.). Eine hohe Genauigkeit beim Zielen ist also nicht erforderlich.

Aber was ist nun der gro"se Unterschied zwischen dem digitalen und dem Traverso-Weg? Unserer Erfahrung nach vermittelt die Bedienung eines PCs mit der Maus immer ein Gefühl von ,,Fernsteuerung''. Man bewegt die Maus auf der Tischplatte, und steuert damit den Zeiger auf dem Bildschirm. Ob sich der Zeiger auf einem Bedienelement befindet oder nicht erkennt man nur visuell. Dies erfordert ein hohes Ma"s an Zielgenauigkeit und führt zu vielen Fehlklicks. Einen echten Knopf direkt mit den Fingern zu drücken ist dagegen ergonomischer und intuitiver, und da man den Knopf mit den Fingern ertasten kann, ist oft nicht einmal eine besondere Aufmerksamkeit notwendig um ihn zu finden oder um zu wissen, wie weit er gedreht wurde. In Traveso versuchten wir deshalb ein Konzept zu finden, das es erlaubt, Funktionen über echte Knöpfe zu erreichen, ohne spezielle Hardware-Erweiterungen anschaffen zu müssen. Was liegt also näher als die Tastatur des PCs dazu zu verwenden? Die Maus wird lediglich dazu verwendet, das Objekt, das die Funktion empfangen soll, zu identifizieren. Die Funktion selbst wird dann von der linken Hand auf der Tastatur ausgeführt. Zu allem Überfluss können wir die meisten der 104 Tasten sogar mehrfach mit Funktionen belegen, was dem Benutzer direkten Zugriff auf eine riesige Anzahl Operationen erlaubt. Das Konzept ist eng verwandt mit der Bedienung von einigen Computerspielen (z.\,B. ,,First-person shooters''), die ein Maximum an Effizienz und direktem Zugriff auf Funktionen (gehen, rennen, schie"sen, ducken\dots) und Ausrüstungsgegenstände erfordern. Zusammengefasst bietet das Konzept der weichen Selektion also folgende Vorteile:

\begin{itemize}
 \item Die Distanzen zwischen den Tasten auf der Tastatur sind meist kürzer als zwischen den Knöpfen auf dem Bildschirm.
 \item Die Trefferquote für eine bestimmte Taste ist höher als für einen Knopf auf dem Bildschirm.
 \item Die Verwendung beider Hände ermöglicht eine schnellere Bedienung, entlastet die Maus-Hand, und führt zu einer weniger ermüdenden Arbeitsweise.
 \item Die Bedienung fühlt sich ,,direkter'' an, das Gefühl der ,,Fernbedienung'' wird reduziert.
 \item Direkter Zugriff auf viele Funktionen ermöglicht weniger und flachere Menüstrukturen.
 \item Die Tasten können blind gefunden werden, wodurch man schneller arbeiten kann und weniger Details auf dem Bildschirm erkennen muss.
\end{itemize}

Diese Vorteile erkauft man sich jedoch mit einer deutlich steileren Lernkurve, besonders ganz am Anfang. Da die Tasten nicht mit ihrer Funktion angeschrieben sind, muss man sie auswendig lernen. Eine sinnvolle Tastaturbelegung, die sich einerseits am Buchstaben der Taste orientiert, und zudem bemüht ist, ähnliche Funktionen möglichst immer auf ein- und dieselbe Taste zu legen, erleichtert jedoch das auswendig lernen, und schon nach kurzer Zeit werden die wichtigsten Funktionen genauso automatisch ausgeführt wie ,,Strg+C'' für ,,kopieren'' und ,,Strg+V'' für ,,einfügen''.

