Traverso is a multitrack audio recording and editing program for GNU/Linux with special emphasis on an intuitive, clean, and above all efficient user interface. The program currently supports recording of any number of audio tracks (only limited by hardware capabilities), basic mixing features, writing to CD, and rendering of the project into various standard audio file formats. The audio engine uses 32 bit floating point precision for all calculations to preserve the highest possible audio quality even after extensive processing.

The user interface uses a contextual interaction concept; instead of relying on the mouse to operate on graphical objects, combinations of mouse and keyboard are used to control the program. This results in a higher flexibility and much faster control of the program when compared to the traditional mouse-based approach. It even goes far beyond the possibilities offered by conventional key shortcuts. The mouse only has to be placed on an object and all functions become available instantly by pressing a key on the keyboard. Since the object under the mouse cursor is automatically selected, this concept is called ``soft selection''.

\section{License}
Traverso is free software; you can redistribute it and/or modify it under the terms of the GNU General Public License as published by the Free Software Foundation; either version 2 of the License, or (at your option) any later version.

This program is distributed in the hope that it will be useful, but WITHOUT ANY WARRANTY; without even the implied warranty of MERCHANTABILITY or FITNESS FOR A PARTICULAR PURPOSE.  See the GNU General Public License for more details.

You should have received a copy of the GNU General Public License along with this program; if not, write to the Free Software Foundation, Inc., 51 Franklin Street, Fifth Floor, Boston, MA  02110-1301, USA.

\section{Motivation}
One of the motivations to introduce the concept of soft selection was our belief in the superiority over the traditional point-and-click concept in regard to efficiency, speed, and ergonomics. The full potential of soft selections develops after an initial learning phase, because something that breaks with existing standards requires time and effort to adopt, and it takes even more time to overcome long-trained habits. But let's have a look at the different working styles with a trivial example. Suppose we want to do something as simple as switching ``Solo'' or ``Mute'' of a track on and off.

\minisec{The ``analogue way''}
In an analogue recording studio with a mixing desk, you have lots of channels, and each channel has lots of buttons, faders, knobs etc. To toggle ``Solo'' or ``Mute'', you identify the channel strip on the mixing desk, and press the corresponding button. If you want to do the same with several channels, you can quickly press many buttons in a row. The fact that there is a dedicated button for each and every function (which is not always trivial to find, depending on the size of the desk), makes it easy to switch the button, but results in mixing desks being huge and complicated electronic devices.

\minisec{The ``Digital Way''}
On a conventional digital audio workstation (DAW) you identify your channel and press the corresponding solo or mute button with the mouse. Depending on the user's skills and the size of the button and screen, this can already be a minor challenge. Switching several channels in a row, however, is very slow and inefficient, because hitting the button requires careful positioning of the mouse cursor each time.

\minisec{The ``Traverso Way'': Soft Selection}
By using additional input devices allowing to control the DAW software with real faders and buttons combines the advantages of digital and analogue systems. With its concept of ``soft selections'' Traverso follows a similar approach without the need for dedicated hardware.

In Traverso you move the mouse over the track you want to process, and press a key on the keyboard, e.g. ``U'' for mute, or ``O'' for solo. Most users hit the key without looking at the keyboard. The track panel is a large area, which is entirely sensitive for the key actions, so even if several tracks should be switched, the mouse cursor can be placed anywhere on the track, requiring much less aiming.

So what is the difference between the digital and the Traverso way? From our experience, moving the mouse on the desk in order to move the cursor on the screen is like using a remote control. The mouse is moved on the desk to move the cursor on the screen. Whether or not the cursor is placed on a control element an only be determined visually. Pressing buttons with our fingers, one for mute, one for solo etc., is more ergonomic and intuitive. Real buttons can be found without special attention and we can feel how much the fader was moved without looking at it. In Traverso we tried to implement an interface concept which gives a feeling as direct as analogue buttons, without requiring additional hardware control devices. So once we have identified the channel we want to process (by hovering the mouse cursor over it), we want a real world button for as many functions as possible. And since our left hand could as well be ready on the keyboard instead of picking our nose, we have all the 104 buttons of our keyboard at our disposal, which can give direct access to a large number of actions. This concept is closely related to the handling of action games (e.\,g. ``first-person shooters''), which are highly optimized for efficient and intuitive access to various actions (moving, running, shooting, ducking, \dots) and a large number of tools. To summarize, the advantages of soft selections are:

\begin{itemize}
 \item the distances on the keyboard are shorter than on the screen
 \item the hit-to-miss ratio for key strokes is much better than for buttons on the screen
 \item using both hands allows to work faster and relieves the mouse hand, resulting in a less fatiguing working style
 \item the ``remote control'' feeling is reduced
 \item more actions can be reached directly, requiring less and shallower menu structures
 \item the keys can be found blindly, leading to less distraction from the work flow
\end{itemize}

The downside is a steeper learning curve, particularly at the very beginning, since the keys are not labeled with the function name (``Solo'', ``Mute'', ``Rec'' etc.). But you will soon internalise the commands just as you did with ``Ctrl+C'' for copy and ``Ctrl+V'' for paste.
