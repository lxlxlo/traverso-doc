Cuando conocemos a alguien, la decisión de si esa persona nos caerá bien o no, se realiza en los primeros segundos. Y con más razón si esa persona es, digamos, diferente. Es decir, si en algún sentido no encaja en lo que nosotros entendemos por una persona normal, ya sea por su extraño comportamiento, su pelo rosa y sus vaqueros rotos, su acento, o su impresionante belleza. La primera impresión siempre es decisiva.

Aunque Traverso no es una persona, \emph{es} bastante distinto de otros programas, y si un usuario pretende usarlo como usaría otros programas, seguramente se sentirá frustrado. Para evitar la desilusión y hacer que la primera impresión sea positiva, recomendamos por lo menos leer el primer capítulo de este libro, que explica lo básico acerca de la interfaz de usuario. Ser diferente es bueno a veces, pero en este caso se requerirá que Ud. aprenda un par de cosas al principio. Si no, no será capaz de hacer funcionar el programa ni explorarlo por sí mismo. Pero no se desanime, aprender lo básico sólo lleva un par de minutos, y tras algunas horas o días lo habrá asumido y descubierto su eficiencia y su potencial. Por tanto, permítanos presentarle un programa que es un poco especial, pero encantador\dots
